% Options for packages loaded elsewhere
\PassOptionsToPackage{unicode}{hyperref}
\PassOptionsToPackage{hyphens}{url}
%
\documentclass[
]{article}
\usepackage{amsmath,amssymb}
\usepackage{iftex}
\ifPDFTeX
  \usepackage[T1]{fontenc}
  \usepackage[utf8]{inputenc}
  \usepackage{textcomp} % provide euro and other symbols
\else % if luatex or xetex
  \usepackage{unicode-math} % this also loads fontspec
  \defaultfontfeatures{Scale=MatchLowercase}
  \defaultfontfeatures[\rmfamily]{Ligatures=TeX,Scale=1}
\fi
\usepackage{lmodern}
\ifPDFTeX\else
  % xetex/luatex font selection
\fi
% Use upquote if available, for straight quotes in verbatim environments
\IfFileExists{upquote.sty}{\usepackage{upquote}}{}
\IfFileExists{microtype.sty}{% use microtype if available
  \usepackage[]{microtype}
  \UseMicrotypeSet[protrusion]{basicmath} % disable protrusion for tt fonts
}{}
\makeatletter
\@ifundefined{KOMAClassName}{% if non-KOMA class
  \IfFileExists{parskip.sty}{%
    \usepackage{parskip}
  }{% else
    \setlength{\parindent}{0pt}
    \setlength{\parskip}{6pt plus 2pt minus 1pt}}
}{% if KOMA class
  \KOMAoptions{parskip=half}}
\makeatother
\usepackage{xcolor}
\setlength{\emergencystretch}{3em} % prevent overfull lines
\providecommand{\tightlist}{%
  \setlength{\itemsep}{0pt}\setlength{\parskip}{0pt}}
\setcounter{secnumdepth}{-\maxdimen} % remove section numbering
\ifLuaTeX
  \usepackage{selnolig}  % disable illegal ligatures
\fi
\IfFileExists{bookmark.sty}{\usepackage{bookmark}}{\usepackage{hyperref}}
\IfFileExists{xurl.sty}{\usepackage{xurl}}{} % add URL line breaks if available
\urlstyle{same}
\hypersetup{
  pdftitle={Verschlüsselung},
  hidelinks,
  pdfcreator={LaTeX via pandoc}}

\title{Verschlüsselung}
\author{}
\date{}

\begin{document}
\maketitle

\paragraph{Symmetrische
Verschlüsselung}\label{symmetrische-verschluxfcsselung}

\emph{Die symmetrische Verschlüsselung ist ein Begriff der Kryptografie.
Damit ist ein Verschlüsselungsverfahren~ gemeint, bei dem für die Ver-
und Entschlüsselung einer Nachricht \textbf{derselbe Schlüssel} benötigt
wird. Dazu muss vorher der gemeinsame Austausch des Schlüssels
vorgenommen werden}

{Encryption\_Symetrical.drawio.svg}\\
Bob hat eine wichtige Information die er Alice, über einen unsicheren
Kanal (zum Beispiel das Web), zusenden möchte. Um sicher zu gehen, dass
die Information nicht von einem dritten gelesen wird (Eve), muss er die
Nachricht zuerst verschlüsseln. Dazu müssen Bob und Alice sich zuerst
auf einen gemeinsamen Schlüssel K1 geeinigt haben. Mit dem Schlüssel K1
verschlüsselt Bob nun die Information und sendet diese über den
unsicheren Kanal an Alice. Auch wenn ein Eve die verschlüsselte
Information mit lauscht, kann er mit der Information nichts mehr
anfangen, da diese durch die Verschlüsselung keinen Sin mehr ergibt. Um
die Information wieder zu entschlüsseln, wird wieder derselbe Schlüssel
K1 benötigt der auch für die für die Verschlüsslung benutzt wurde. Da
nur Alice und Bob zuvor den Schlüssel ausgetauscht haben, können nur sie
den Originaltext lesen.

\hfill\break

\textbf{Vor- und Nachteile}

\strut \\
\textless sodipodi:namedview\\
id="namedview1"\\
pagecolor="\#ffffff"\\
bordercolor="\#000000"\\
borderopacity="0.25"\\
inkscape:showpageshadow="2"\\
inkscape:pageopacity="0.0"\\
inkscape:pagecheckerboard="0"\\
inkscape:deskcolor="\#d1d1d1"\\
inkscape:document-units="mm" /\textgreater{}\\

\begin{center}\rule{0.5\linewidth}{0.5pt}\end{center}

\paragraph{Asymmetrische
Verschlüsselung}\label{asymmetrische-verschluxfcsselung}

\emph{Die asymmetrische Verschlüsselung ist ein Begriff der
Kryptografie. Dieses Verschlüsselungsverfahren arbeitet mit
Schlüsselpaaren. Ein Schlüssel ist der öffentliche (Public Key) und der
andere der private Schlüssel (Private Key). Daten, die mit dem
öffentlichen Schlüssel verschlüsselt wurden, können nur noch mit dem
privaten Schlüssel entschlüsselt werden. Will der Sender eine
Information verschlüsselt an den Empfänger senden, benötigt er den
öffentlichen Schlüssel des Empfängers}

{Encryption\_asymetrical.drawio.svg}

Bob und Alice möchten über einen unsicheren Kanal (zum Beispiel das Web)
Nachrichten austauschen. Um sicher zu stellen, dass kein dritter (Eve)
die Nachrichten mit horcht, entscheiden sich Bob und Alice die
Nachrichten mit der asymmetrischen Verschlüsslung zu Chiffrieren. Dazu
benötigt Bob den öffentlichen Schlüssel von Alice. Der öffentliche
Schlüssel ist für alle sichtbar (auch für Eve). Die Veröffentlichung
kann zum Beispiel über E-Mail oder einen öffentlichen Server passieren.
Mit dem öffentlichen Schlüssel von Alice kann jeder seine Nachricht
verschlüsseln und das Ergebnis an Alice senden. Jedoch kann nur Alice
die Nachricht, mit dem privaten Schlüssel, wieder entschlüsseln.

\hfill\break

\textbf{Vor- und Nachteile}

\strut \\
\textless sodipodi:namedview\\
id="namedview1"\\
pagecolor="\#ffffff"\\
bordercolor="\#000000"\\
borderopacity="0.25"\\
inkscape:showpageshadow="2"\\
inkscape:pageopacity="0.0"\\
inkscape:pagecheckerboard="0"\\
inkscape:deskcolor="\#d1d1d1"\\
inkscape:document-units="mm" /\textgreater{}\\

\end{document}
